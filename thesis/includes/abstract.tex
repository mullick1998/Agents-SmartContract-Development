% Confirmation
\ifx\doclanguage\english
\chapter*{Abstract}

   Blockchain use has skyrocketed in recent years, and various blockchains customized to specific use cases have emerged. The blockchain technology stack has grown tremendously since Bitcoin, the first true blockchain application. One of the most essential blockchain uses is smart contracts. Most present smart contract systems presume that when contracts are executed over a network of decentralized nodes, the majority's decision can be trusted. However, we have seen that people associated with a smart contract may intentionally take steps to manipulate contract execution in order to improve their own gains. To solve this issue, we suggest an agent model as the underlying mechanism for contract execution over a network of decentralized nodes and a public ledger, and we analyze the prospect of prohibiting users from influencing smart contract execution.
    
    Numerous research and application sectors are being impacted by the blockchain technology and idea, and as a result, many see this as an opportunity to find new solutions to old issues or reap innovative advantages. Several writers in the agent community are proposing their own mix of agent-oriented technology with blockchain to address both old and new difficulties. This thesis paper aims to define the prospects, factors to consider, and analyzing information about the content for combining agents with blockchain.
    
    