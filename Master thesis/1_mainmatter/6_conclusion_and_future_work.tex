\iflanguage{ngerman}
{\chapter{Fazit und Ausblick}}
{\chapter{Conclusion and Further Work}}

\label{sec:conclusion}

In this thesis, we discussed about the multiple-agent system and blockchain technology, which are well-known technologies that will likely be employed in the future in a significant supply chain sector. We provide an implementation regarding the creation of smart contracts with JASON \ac{BDI} agents, whereby the smart contracts are constructed using Solidity, the agents are developed with the JASON run on Python-based interpreter, and both are integrated with the web3 package. 

\vspace{.5cm}

Our \textit{first research question} was answered by our implementation, and we arrived at the conclusion that roles for retailer, wholesaler, and manufacturer can be represented in a Solidity-based supply chain application with Agent \ac{BDI} Library. The \ac{BDI} Agent Library can be used to implement \ac{MAS} on the blockchain in addition to implementing the beliefs, desires, and intents of the agents, which can involve a variety of actions and behaviors in the supply chain.

\vspace{.5cm}

We observed that agents can coordinate their actions with rules using smart contracts and that answers our \textit{second research question}. The rules that control how supply chain participants interact may be referred to as smart contracts. These regulations may specify the terms and circumstances for carrying out certain activities, such as when a retailer may buy products from a wholesaler or when a manufacturer may send goods to a store. By using the smart contract's features and providing the necessary data and settings, agents can communicate with it. The smart contract may then put the rules into action and determine how the interaction will turn out based on the state of the supply chain at the moment and the data provided by the agents.

\vspace{.5cm}

Additionally, smart contracts can be used to create consensus and dispute resolution processes, such as when a contract or agreement is not fulfilled by all parties or when there are disagreements. Further, the smart contract may be designed to encourage automated, transparent communication and coordination amongst supply chain agents. All parties in the supply chain may maintain tabs on the movement of goods and adjust their operations accordingly by using smart contracts, for instance, to automatically update inventory levels and shipment status of products.

\vspace{.5cm}

In relation to our \textit{third research question}, we determined that synchronizing AgentSpeak tactics and goals with on-chain smart contract transaction payloads is achievable, albeit the precise implementation would rely on the AgentSpeak library and the blockchain platform employed. The technique we adopted was to utilize the AgentSpeak library to specify the agents' decision-making strategies and goals, and then use smart contracts to manage transaction execution based on those strategies and goals. It ought to be noted that synchronizing AgentSpeak tactics and goals with on-chain smart contract transaction payloads is a difficult operation that would need a thorough grasp of both AgentSpeak and blockchain technology.
\vspace{.5cm}

As part of our ongoing efforts, we may expand the functionality of our program and create a fantastic user interface, which will allow us to modify the code and make it usable. In order to utilize the web3j package, we also anticipate having a successful implementation with our other choices, maybe following a significant update in all the other languages we tried to employ that include a Java-based interpreter. Future work on integrating smart contracts with \ac{BDI} agents will be done in a number of domains, including:

\begin{itemize}
    \item \textbf{Better reasoning and decision-making abilities} \\ 
    One of the most important advantages of \ac{BDI} agents is their ability to make decentralized, independent decisions based on their beliefs, wants, and intents. However, existing smart contract implementations of \ac{BDI} agents may not be as complex in decision-making as anticipated. Future work in this area might include developing and implementing more complex reasoning and decision-making algorithms for \ac{BDI} agents to allow improved supply chain decision-making.

    \vspace{.5cm}
    
    \item \textbf{Better communication and coordination} \\ 
    \ac{BDI} agents can be programmed to interact and collaborate with other agents to achieve shared objectives. However, present communication and coordination channels may fall short of expectations. Future work in this area might include investigating and implementing more advanced communication and coordination methods to increase the efficacy of \ac{BDI} agents in the supply chain.

    \vspace{.5cm}
    
    \item \textbf{Merging with additional \ac{AI} technologies}\\ 
    \ac{BDI} agents can be combined with other \ac{AI} technologies such as machine learning and natural language processing. Future work in this area might include investigating and developing methods for combining \ac{BDI} agents with other \ac{AI} technologies to increase their performance and capabilities.

    \vspace{.5cm}
    
    \item \textbf{Decentralized trustworthiness and reputation} \\ 
    Agents of the \ac{BDI} can be designed to consider the standing and reliability of other agents while making judgments. The methods in place to measure reputation and trust may not be as sophisticated as they may be. Future study in this area may focus on developing and utilizing more sophisticated reputation and trust measurement techniques that can be utilized to increase the efficiency of \ac{BDI} agents in the supply chain.
    
    \vspace{.5cm}
    
    \item \textbf{More advanced simulation ecosystems} \\ 
    Agents' simulation surroundings must be upgraded when they are implemented in increasingly realistic and complicated contexts. Future work in this field might include developing and deploying more realistic and complicated simulation environments for testing and evaluating \ac{BDI} agents in real-world circumstances.

    \vspace{.5cm}
\end{itemize}
Additionally, \ac{BDI} agents are frequently intended to interact with the digital environment and may have limited interaction skills with the real world. Future work in this field might include investigating and implementing solutions for improved connection with the physical environment, such as combining \ac{BDI} agents with IoT devices to enable more effective and efficient supply chain management.