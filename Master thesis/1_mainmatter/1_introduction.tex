\iflanguage{ngerman}
{\chapter{Einleitung}}
{\chapter{Introduction}}

\label{sec:introduction}


The term \textit{"agent"} is commonly used these days. This applies to both \ac{AI} and fields outside of it, such as databases and automated manufacturing. Despite growing in popularity, the phrase has been employed in so many different contexts that it has lost all meaning when not connected to a certain understanding of agent-hood. The terminology "agent" is most frequently used in the context of  \ac{AI} to describe an object that operates constantly and autonomously in a context that also contains other processes and agents. Although the emphasis is on the word "autonomy,"  the definition of autonomy is vague, but it is generally understood to suggest that the behaviors of the agents do not need ongoing human direction or interference, and presumptions are frequently made about the environment of the agents, such as that it is mostly unpredictable.

\vspace{.5cm}

Agents are assumed to be robotic agents, in which case additional aspects including sensory input, motor control, and time constraints are stated. Finally, agents are frequently perceived as "high-level." Although this definition is rather ambiguous, many people use it to distinguish agents from other bits of hardware or software. Symbolic representation and/or cognitive-like capabilities are examples of "high-level" abilities. Agents can be "informable," have symbolic planning as well as stimulus-response rules, and even talk. Because this notion is not commonly understood in \ac{AI}, the concept of agenthood in \ac{AI} is hazy. As a result, when we use the term "agent," we need to be clear about what we mean, which is as follows: An agent is an entity whose state is regarded to be formed of mental aspects such as beliefs, capabilities, decisions, and commitments. These elements are precisely specified and roughly correspond to their equivalents in common sense. Therefore, according to this perspective, an agent only exists in the programmer's head. Any hardware or software component only qualifies as an agent if its analysis and management have been done in this way.

\vspace{.5cm}

Recent years have seen a surge in interest in the study of computational agents with the ability to behave rationally. In the commercial world, the architecture of the systems needed to carry out high-level management and control operations in complex, dynamic circumstances is becoming more and more crucial. Examples of these systems include those used in air traffic control, telecommunications networks, business operations, spacecraft, and healthcare. Such systems are exceedingly difficult and expensive to develop, validate, and maintain when utilizing typical software approaches. Agent-oriented systems, which are based on a fundamentally different perspective on computational entities, provide opportunities for a qualitative shift in this stance. 

\vspace{.5cm}

One such architecture envisions the system as a sensible agent with distinct mental attitudes of \ac{BDI}, reflecting the informative, motivational, and deliberative states of the agent, respectively. When deliberation is subject to resource constraints, these mental attitudes govern the system's behavior and are crucial for obtaining adequate or optimal performance.

\vspace{.5cm}

\ac{BDI} agents can be considered for both a formulation of a theory and a practical design standpoint. \textit{AgentSpeak(L)} is a generalization of one of the implemented \ac{BDI} systems that allow agent programs to be developed and processed in the same way as a horn-clause rationale program. It is a language of programming based on a limited first-order language with events and actions. The programs created in AgentSpeak(L) govern the agent's behavior (i.e., its interaction with the environment). The agent's beliefs, desires, and intentions are not explicitly stated as modal formulae. Instead, as designers, one may attribute these ideas to AgentSpeak(L) agents. The agent's present belief state may be considered a model of itself, its environment, and other agents. The states that the agent seeks to bring about based on external or internal stimuli can be viewed as desires, and the adoption of programs to meet such stimuli can be viewed as intents.

\vspace{.5cm}

It is therefore easy to design an agent capable of controlling its surroundings on its own, but this becomes troublesome when the environment is shared by more than one agent. AgentSpeak(L) is extended by languages such as Jason and ASTRA. They could be used to implement that language's operational semantics and provide a foundation for the development of \ac{MAS} with different user-customizable features.

\vspace{.5cm}

The implementation of these agents in specialized fields would be helpful when discussing the agents' ability to communicate using AgentSpeak(L). One of the most significant areas will be supply chain management, where we may assign roles to various agents such as manufacturers, wholesalers, and retailers. An adaptable supply chain allows retailers, distributors, and manufacturer agents to adjust to environmental changes. Each agent has a core objective as well as related secondary objectives. \ac{BDI} agents can be used to improve decision-making in supply chain management by allowing for the simulation and optimization of different scenarios and strategies. For example, a \ac{BDI} agent could be used to forecast demand for a product and adjust inventory levels accordingly. 

\vspace{.5cm}

Additionally, \ac{BDI} agents can also be used for negotiation and coordination with other agents to obtain better prices or delivery schedules. Despite what might be expected, there will always be a transaction anytime agents engage in the supply chain, which must be documented. Any centralized database could be utilized by agents to store the transaction, but is that the best option? A programmable ledger can be used to record the movement and ownership of items between each agent. So rather than keeping these interactions in the form of transactions in databases or spreadsheets, it will be perfect for storing them in smart contracts, which can be inspected for tracking or future references.	

\vspace{.5cm}

The concept of \textit{smart contracts}, or computer protocols intended to automatically facilitate, authenticate, and implement the negotiation and application of digital contracts without the need for centralized authorities, has been revived in recent years as a result of the rapid development of cryptocurrencies and the \ac{BCT} that underpins them. Smart contracts have been incorporated into popular blockchain-based development platforms like Ethereum and Hyperledger. They have a wide range of potential application areas in the globalized era and intelligent sectors of the economy, including the financial sector, management, healthcare, and \ac{IoT}, among many others. Smart contracts offer a tamper-proof and auditable record of all the transactions between various parties, which may be used to improve traceability and transparency in the supply chain. In lengthy and complicated supply chains, retailers and distributors follow the items' flow, which might be very helpful.

\vspace{.5cm}
While conducting a further in-depth study on programming agents and smart contracts, we would want to investigate the viability of an interaction mechanism between agent-oriented programming and smart contracts in this thesis. We shall attempt to address the research questions (RQ) listed below.
\begin{itemize}[label={}]

\item \textbf{RQ1.} What character do we need in smart contracts to coordinate with the beliefs, desires, and intentions of agents? \\

 \item \textbf{RQ2.} How to represent supply chain roles (manufacturer, wholesaler, retailer) in Solidity language such that \ac{BDI} model agent library can coordinate and synchronize AgentSpeak plans and goals with on-chain smart contract transaction payload? \\
 
 
\end{itemize}

In this thesis, \textbf{\textit{Smart contract development with Jason \ac{BDI} Agents}}, we will attempt to develop an application that will display various agents, allowing agents across the supply chain to interact with one another while comparing different \ac{BDI} agent Framework to answer the above questions. This ecosystem will allow agents to maintain relationships, store transactions using smart contracts, and determine whether both technologies can be used to create a fair and efficient dispute resolution mechanism in the supply chain by providing a transparent and tamper-proof record of the agreements and interactions between parties.

\section{Motivation}

The blockchain is increasing influence on various research and application fields, from distributed computing and storage to supply chain management and healthcare accountability. There are many distinct use cases where the promise of supplying a secure and fault-tolerant ledger that mutually untrusted parties may use to monitor computations and interactions completely dispersed and decentralized without the need for a central authority is intriguing. 

\vspace{.5cm}

The agent community is no different; it began expressing interest in the opportunities presented by the blockchain to either solve long-standing problems (such as trust management in open systems, and accountability of actions for liability, among others) or to take advantage of expected benefits to endow a system with new properties (such as novel infrastructures for \ac{MAS}, trustworthy coordination). \ac{BDI} agents can be used to build more resilient and sustainable supply chains by simulating and optimizing various sustainability scenarios, such as examining the environmental effect of different transportation routes or manufacturing processes. \ac{BDI} agents can also be programmed to automatically initiate smart contracts with suppliers or customers to enforce sustainability-related commitments.

\vspace{.5cm}

Smart contracts play a critical role in enabling the blockchain to function as a general-purpose distributed computing engine while retaining its core properties of security and trust, and they actually extend their reach to cover computations other than data storage and management. Smart contracts can be utilized to improve supply chain traceability and transparency by providing a tamper-proof and auditable record of all transactions between multiple parties. This is especially beneficial in complicated and extensive supply chains where it is difficult to maintain track of products' movements.

\vspace{.5cm}
The convergence of the agent and blockchain worlds appears to be promising: Agents are distributed autonomous entities whose interactions should be managed and mediated in trustworthy ways; blockchains and smart contracts, on the other hand, are trust-sensitive mechanisms for mediating and managing interactions. Combining \ac{MAS} with smart contracts can have several potential benefits for supply chain management and other industries:

\begin{itemize}
    
\item \textbf{Improved coordination} \\ The supply chain can be made more efficient and successful by coordinating the operations of numerous actors using smart contracts. Smart contracts could be used to set the rules and restrictions that govern interactions between supply chain agents, which can assist to coordinate the operations of multiple agents and limit the risk of mistakes or inconsistencies.

\vspace{.5cm}

\item \textbf{Decentralized decision-making} \\ By programming agents to make decisions based on their beliefs, desires, and intentions, multiple agent systems can provide a decentralized and autonomous approach to decision-making. This allows agents to make decisions based on the most current information and adapt to changing conditions in the supply chain.

\vspace{.5cm}

\item \textbf{Transparency} \\ Smart contracts and multiple agent systems can provide transparency in the supply chain. Smart contracts can be used to record all activities and interactions among agents, allowing all parties to see the flow of goods and the state of the supply chain.

\vspace{.5cm}

\item \textbf{Improved security} \\ By using smart contracts to define the rules and constraints that govern the interactions between agents, the system can be more resistant to malicious attacks and fraud. Additionally, multiple agents can work together to detect and defend against any malicious attempt.

\vspace{.5cm}

\item \textbf{Scalability} \\ Smart contracts can be used to handle a large number of transactions and interactions between agents, which can help scale the system as the number of agents and transactions increases.

\vspace{.5cm}

\item \textbf{Interoperability} \\ By using smart contracts to facilitate communication and coordination among multiple agent systems, the system can be designed to be interoperable with other blockchain and non-blockchain platforms.

\vspace{.5cm}

\item \textbf{Flexibility} \\ By using multiple agents with \ac{BDI} approach, the system can adapt to changing conditions and goals, making the system more flexible.
\end{itemize}

It is crucial to remember that while merging \ac{MAS} with smart contracts might have numerous advantages, doing so requires knowledge of both disciplines, and the system's design must take into consideration the supply chain's complexity and dynamic nature.

\section{Our contributions}

The contribution of the paper are as follows:

\begin{itemize}
    \item We are addressing the literature study on \ac{BDI}-based agent-oriented programming and autonomous agent-based decentralized smart contract execution.

    \item We demonstrate and analyze the aspects of \ac{MAS}-\ac{BCT} created by integrating agents with smart contracts, as well as the implications for agent-oriented practice and blockchain. The primary goal was to utilize AgentSpeak(L) to program agents and store their interactions using smart contracts.
    
\vspace{.5cm}

    \item  The fundamental concept is to construct \ac{MAS} using Jason, an extension of AgentSpeak(L) as well as comparing with other \ac{BDI} agent frameworks and finally creating smart contracts with Solidity and integrating them altogether.
    
\vspace{.5cm}

    \item Our research enabled the usage of Jason with both the Python interpreter and Java. As an alternative to Jason, we also tested using ASTRA. Because Java was the prior implementation language, ASTRA's type system is based on it. By using a comparable type system, translating between ASTRA and Java is made easier and more clear. Vyper was also taken under consideration for Smart contracts while switching from Java to Python, but was later abandoned owing to some of its drawbacks.
    
\vspace{.5cm}

\end{itemize}

In order to integrate them, we conducted extensive study while doing that. When choosing a programming language to integrate agents written in agentSpeak(L) with smart Contracts written in Solidity, we run into a myriad of challenges.

\section{State of the Art}

Some research reveals methods for connecting the \ac{MAS} and \ac{BCT} by putting agents and blockchains side by side, enabling agents to utilize blockchain services as needed. Current model research on \ac{BDI} agents and smart contracts is on leveraging \ac{BDI} concepts to create and build optimization techniques that can communicate with blockchain-based smart contract systems. This involves the integration of \ac{BDI} agents with blockchain-based decision-making processes, such as consensus protocols and smart contract execution, as well as the application of \ac{BDI}-based reasoning to manage uncertainty and adapt to changing situations in decentralized contexts.

\vspace{.5cm}
Use of \ac{BDI} agents for decentralized autonomous organizations (DAOs), which are decentralized, self-governing organizations that run on smart contracts built on the blockchain, is one illustration of this. A DAO's members can be represented by \ac{BDI} agents, who can then employ \ac{BDI}-based reasoning to decide on their behalf. Using \ac{BDI} agents for smart home automation is another field of research; these agents can operate as a bridge between users and gadgets and can make choices based on the users' intentions, beliefs, and desires.

\vspace{.5cm}

Overall, the present state of the art for \ac{BDI} agents and smart contracts is still in its early stages; however, with growing interest in blockchain technology, research, and development in this subject is expected to expand in the future years. This thesis will primarily focus on agent-oriented models based on the \ac{BDI} framework and connected to the blockchain for the goal of running a supply chain utilizing agents. In Chapter 2, we dug further into the subject by asking some Literature review questions (LRQ) to gain a better understanding.

\section{Evaluation Results}

In order to answer our research questions, our evaluation was conducted using three separate criterias: (1) Choosing among various \ac{AOP} Language with \ac{BDI} framework, (2) Smart Contract Functionality, and (3) Outcomes of combining smart contracts with Agents. Experimental findings from these many angles are systematically examined.

\subsubsection{Choosing \ac{AOP} Language with \ac{BDI} Framework}

  In order to understand how they operate and determine whether they are compatible with other languages to be used to include them into blockbhain, we developed the \texttt{.asl} files using Jason and ASTRA. We utilized Java and Python based interpreters to create \ac{MAS} utilizing Jason AgentSpeak(L), and we also tested ASTRA to learn more about the differences between the languages and see if it works well with Java and makes use of all Java packages like \texttt{org.web3j}. 

  \vspace{.5cm}

  Our analysis indicated that, while the code in the \texttt{.asl} files for Jason's Java and Python-based interpreters is similar, the two employ different commands to communicate. The main distinction between the two is that the Java-based interpreter requires a \texttt{mas2j} file to execute \ac{MAS}, but the Python-based interpreter only requires a python script that creates an environment for several agents to communicate with each other.
  \vspace{.5cm}

  Although ASTRA-written agents are somewhat unique since they are more likely to resemble Java-style syntax. There is no need to create separate files for the interaction between agents because all the agents may be written in a single file with the \texttt{.astra} extension.
  
\subsubsection{Smart Contract Functionality}
  
  We will investigate if it is possible to describe supply chain members (such as a retailer, wholesaler, and manufacturer) in Solidity, the programming language used for writing smart contracts on the Ethereum blockchain. We examined h,ow well smart contracts run on severbasedtworks, includareg the local network, Infura utilizing Rinkeby, and Alfajores, as well as how long it takes to construct each contract and Java-basedption amount. Local networks consistently had the lowest delay time acrosPython-basedtworks after testing the prototype for smart contracts, thus we utilized this network to test our subsequent scenarios involving agents. Since CELO is simpler to obtain than RinkenyETH, testing on Alfajores test network was likewise simpler than on Rinkeby test network. In none of our test scenarios did we use the mainnet.
  
  \vspace{.5cm}
  
  Along with that, we also stated reason and provided information on why Solidity is preferred over Vyper. Vyper, on the other hand, is built on Python. Solidity is similar to JavaScript, however Vyper lacks several functions that Solidity has, helping to make Solidity more efficient.

\subsubsection{Outcomes of smart contracts with Agents}

The principal goal of this thesis was to operate the agents in a \ac{MAS} while adapting a supply chain and collaborating \ac{BCT} to it in the form of smart contracts. We experimented with combining several AgentSpeak languages with web3 libraries for the assessment. Only \texttt{web3.py} and a Python-based interpreter for Jason were found to be effective for the implementation. With other web3 libraries, such as the \texttt{web3j} and \texttt{web3.js} libraries, we had problems with jar files and missing packages while using \ac{AOP}, since ASTRA and java-based interpreter of Jason can't work with the core libraries required to run smart contracts.

 \vspace{.5cm}
 
In order to make it easier to grasp, the evaluation of the thesis is outlined in more detail later in the chapter.

\section{Outline of the thesis}
\begin{itemize}
    \item The research and associated work that has been done in the area of our study are covered in Chapter 2 with literature research questions(LRQs),
    
\vspace{.5cm}

    \item Technical information needed to comprehend the ideas offered in this thesis is provided in Chapter 3,
    
\vspace{.5cm}

    \item The system's high-level design, architecture, technical details and several sample use cases created for this work are covered  are covered in Chapter 4,
    
\vspace{.5cm}

    \item Chapter 5 details the application's outcomes and evaluation.
    
\vspace{.5cm}

    \item Chapter 6 comprises a self-discussion of the primary concepts identified when researching study subjects,

\vspace{.5cm}

    \item Conclusions of the system's evaluation are summarized in Chapter 7,

\vspace{.5cm}
  
    \item Design enhancements for upcoming and additional study subjects that result from this effort are covered in Chapter 8.
\end{itemize}


