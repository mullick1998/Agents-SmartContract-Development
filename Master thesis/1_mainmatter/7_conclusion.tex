\iflanguage{ngerman}
{\chapter{Fazit und Ausblick}}
{\chapter{Conclusion}}

\label{sec:conclusion}
This study's intended use involved creating a decentralized supply chain network with a variety of agents interacting with one another and smart contracts governing the flow of goods and information. Smart contracts are evaluated separately to test the concept, accounting for the gas cost and the time needed for each contract to deploy and produce on different networks. The Jason framework is used to analyze \ac{BDI} agents in a multi-agent system. Using a local test network, we obtained the required findings from both analyses.
\vspace{.5cm}

According to the findings presented in this thesis, the integration of both technologies can be accomplished by combining a framework based on \ac{BDI} and a library for interacting with the Ethereum blockchain. By combining these two technologies, intelligent agents with the ability to communicate and store interactions as transactions utilizing smart contracts can be created. These technologies automate processes, analyze data, and make decisions based on beliefs, desires, and intentions, which can improve supply chain operations, increase transparency, and reduce costs. These systems' ability to track goods, optimize inventory, and effectively manage resources has the potential to revolutionize supply chain management.

\vspace{.5cm}

In spite of certain drawbacks, such as incompatibilities and varied objectives of both technologies, we discovered that several smart contract features, such as transparency and self-execution, can significantly assist \ac{BDI} agents to align their beliefs, desires, and intentions in the case study of the supply chain. The integration of \ac{BDI}-based multi-agent systems with blockchain technology, notably smart contracts, has been shown in this thesis. The results of this study potentially open the door for the development of new \ac{MAS}-based decentralized applications in a variety of fields.


