\iflanguage{ngerman}
{\chapter{Fazit und Ausblick}}
{\chapter{Conclusion}}

\label{sec:conclusion}

The proposed application of this research was to build a decentralized supply chain network with various agents interacting with one another and smart contracts regulating the exchange of information and products. As a result, the supply chain process would be more transparent, traceable, and effective. To investigate the idea, smart contracts are tested individually, taking into account the gas cost and time required for each contract to deploy and generate on various networks, and \ac{BDI} agents are also evaluated in a multiple-agent system using the Jason framework.

\vspace{.5cm}

According to the research described in this thesis, the merge of both technologies can be achieved by combining a library for interfacing with the Ethereum blockchain and a \ac{BDI}-based framework. Combining these two technologies enables the development of intelligent agents that can communicate and store interactions in the form of transactions using smart contracts. These technologies can enhance supply chain operations, promote transparency, and save costs by automating procedures, evaluating data, and making decisions based on beliefs, desires, and intentions. These systems have the potential to transform supply chain management due to their capacity to track commodities, optimize inventories, and efficiently manage resources.

\vspace{.5cm}

This thesis has shown the possibility of merging \ac{BDI}-based multi-agent systems with blockchain technology, specifically smart contracts. The \ac{BDI} model provides a strong foundation for developing intelligent agents capable of making decisions based on their beliefs, desires, and intentions, while blockchain technology and smart contracts provide a safe and decentralized platform on which these agents can interact. The findings of this study might pave the way for the creation of new decentralized applications and \ac{MAS} in various domains.


