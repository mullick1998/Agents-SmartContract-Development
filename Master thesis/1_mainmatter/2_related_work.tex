\iflanguage{ngerman}
{\chapter{Verwandte Arbeiten}}
{\chapter{Related Work}}
\label{sec:related}


The pertinent research in the field of agent programming using smart contracts will be covered in this part. The associated papers' subjects were arranged to address the following inquiries: 
\begin{itemize}[label={}]
    \item Q1. How to determine the distinction between object-oriented and agent-oriented programming, and how to build agents and \ac{MAS} with JASON, \ac{ASTRA} or other domain-specific \ac{AOP} languages, and have them interact in supply chain management?\\
    
    \item Q2. After understanding \ac{MAS}, how to include smart contracts into \ac{MAS}, record agent interactions on a blockchain using smart contracts, and store transactions on a blockchain?\\

    \item Q3. What's the need for more study on this area if there are existing blockchain-based \ac{MAS} applications in functioning?
\end{itemize}

\section{Assimilating Agents and Agent Oriented Programming}

To address our \textit{first question of this chapter} about learning about \ac{AOP}, \ac{MAS} and other concepts to explore with our thesis, the articles listed below assisted in gaining a comprehensive view on all of the topics and understanding them better.

\begin{itemize}[label={}]
\item \textbf{Paper Title: \textit{Exploring AOP from an OOP perspective}}\\

Researchers in agent-oriented programming have created a number of agent programming languages that successfully connect theory and practice. Unfortunately, despite these languages' popularity inside their own communities, the larger community of software engineers has not found them to be as intriguing. The need to bridge the cognitive gap that exists between the notions behind standard languages and those underpinning \ac{AOP} is one of the key issues facing \ac{AOP} language developers. 
 
 \vspace{.5cm}
 
In this paper \cite{astra}, a conceptual mapping between \ac{OOP} and the AgentSpeak(L) family of \ac{AOP} languages was made in an effort to create such a linkage. This mapping examines how AgentSpeak(L) notions connect to \ac{OOP} principles and the concurrent programming concept of threads. After that, they used the study of this mapping to inform the creation of the \textbf{\ac{ASTRA}} programming language.

 \vspace{.5cm}
 
\item \textbf{Paper Title: \textit{\ac{BDI} Agent Programming in AgentSpeak Using JASON}} \\

This study \cite{jasonBDI} is premised on the instruction that was offered as a part of CLIMA-instructional VI's program. The tutorial's goal was to provide a general overview of the functionality offered by JASON, a \ac{MAS} development platform built around an interpreter for an enhanced version of AgentSpeak. The most well-known and extensively researched architecture for cognitive agents is the \ac{BDI} architecture, and AgentSpeak is a sophisticated, logic-based programming language that was inspired by the \ac{BDI} design.

\vspace{.5cm}

The paper also discusses how agent-based technology has become increasingly popular for a number of reasons, including how well-suited it has proven to be for the invention of a wide range of applications, such as for air traffic control, autonomous spacecraft control, healthcare, and industrial systems control, to mention a few. These are undoubtedly application areas where reliable systems are needed. The fact that formal verification techniques tailored specifically for \ac{MAS} are also an area that is luring much research attention and is likely to have a major impact on the uptake of agent technology is one of the benefits of the approach to programming \ac{MAS} that results from the research reviewed in this paper.

\vspace{.5cm}

\item \textbf{Paper Title: \textit{Domain specific agent-oriented programming language based on the Xtext framework}} \\

One of the most reliable methods for creating distributed systems is the agent technology. A runtime environment that allows the execution of software agents is presented by the multi-agent middleware XJAF. The researchers suggest a domain-specific agent language called \textbf{ALAS}, whose major function is to facilitate the construction and execution of agents across heterogeneous platforms, in order to address the issue of incompatibility. According to the demands and needs of the agents, a metamodel and grammar of the ALAS language have been developed to describe the language's structure. The development of the compiler and the creation of Java executable code that can be run in XJAF are both covered in this paper \cite{xtext}.

\vspace{.5cm}

\item \textbf{Paper Title: \textit{Agent-Oriented Supply-Chain Management}} \\

In order to build an agent-oriented software architecture for controlling the supply chain at the tactical and operational levels, this paper \cite{agSupch} examines problems and offers solutions. The method is based on the employment of an agent building shell, which offers assured, reusable, and generic components. It sees the supply chain as being made up of a group of intelligent software agents, each of which is in charge of one or more supply chain activities. 

\vspace{.5cm}

These agents collaborate with one another to plan and carry out their tasks and provide services for communicative-act-based communication, conversational coordination, role-based organization modeling, and other things. They attempted to demonstrate two nontrivial agent-based supply-chain designs using these elements that might allow intricate cooperative work and the control of disruption brought on by stochastic occurrences in the supply chain.

\vspace{.5cm}

The objective of developing models and methods that allow \ac{MAS} to do coordinated work in practical applications has been advanced by the research in a number of different ways. These strategies are carried out by the agents, which causes several organized dialogues to occur amongst the agents. These concepts have been supported by the development of a useful, application-neutral coordination language that offers tools for describing coordination-enhanced plans as well as the interpreter supporting their execution. The researcher of the study has tested the coordination language and the shell on a number of issues, including supply chain coordination initiatives carried out in collaboration with industry. Despite the fact that the number of solutions they built and the number of users of our system are both limited, the evidence they have so far shows that their methodology is promising in terms of the naturalness of the coordination model, the effectiveness of the representation and power, and the usability of the provided programming tools.
\end{itemize}

\section{Linking Blockchain with Agents}

The principal objective was to include BCT into MAS, which is a very current and innovative subject that may be applied in a futuristic supply chain. Some researchers have already worked on it, which has given us some ideas on how to construct smart contracts with JASON BDI agents. The papers listed below are aligned to our work and can be read to have a stronger insight to answer the \textit{second query we posed in this chapter}.

\begin{itemize}[label={}]

\item \textbf{Paper Title: \textit{From Agents to Blockchain: Stairway to Integration}} \\

The researchers who conducted this study attempted to throw some insight on the most recent integration efforts, mostly from a "agent-vs.-blockchain" perspective. They stressed the possibility of an alternate strategy, which they referred to as "agent-to-blockchain." Finally, they described the "agent-to-blockchain" approach along a pathway upgrading smart contracts towards complete agency, in both dimensions, after acknowledging the presence of two integration dimensions, a computational and an interactional one.

\vspace{.5cm}

In this article \cite{ag2bc}, the researchers provide a roadmap and highlight the challenges that still need to be resolved in order to understand which are the opportunities, the dimensions to take into account, and the different ways available for combining agents and blockchain. They then discussed the case of Tenderfone \cite{tenGlab}, a custom blockchain that offers proactive smart contracts as the initial step along the road-map, equipping smart contracts with control flow encapsulation, re-activeness to time, and asynchronous communication means, as both validation of their road-map and grounds for future development.

\vspace{.5cm}

\item \textbf{Paper Title: \textit{Decentralized Execution of Smart Contracts: Agent Model Perspective and Its Implications}}\\

After close scrutiny, the authors of the paper \cite{decentralized} concluded that users who are connected with a smart contract may deliberately try to influence the contract's execution in order to improve their own advantages. In order to address this issue and discuss the possibility of preventing users from manipulating smart contract execution by using game theory and agent based analysis, the authors of this paper propose an agent model as the underlying mechanism for contract execution over a network of decentralized nodes and public ledger.

\vspace{.5cm}

To simulate the execution of smart contracts over a decentralized network of nodes and participants utilizing blockchain and public ledger, the authors of this study developed an agent-based framework. In contrast to the widely held belief that the results of smart contract execution can be trusted, the agent-based model of smart contract execution makes the assumption that nodes may have an incentive to influence or lie about the results of execution of the contract in exchange for personal benefits or financial gains, even if they are not directly involved in a contract. It had been noted that users who are directly or indirectly involved in a smart contract may behave intelligently to influence the execution outcome of the smart contract. According to the agent-based approach, it might be possible to stop users from cheating when it comes to contract execution or lying about the outcome.

\vspace{.5cm}

It has also been demonstrated that it is realistic to prohibit users from lying about outcomes or manipulating contract execution results if penalties are applied during contract execution and the assumption is made that users are not totally confident in the rationality of other participants. Furthermore, it had been thought that studying irrationality will help us understand how users behave in a decentralized crypto-currency or smart contract system. An important outstanding challenge is the systematic study of irrationality in relation to the execution and consensus of smart contracts. If it is feasible to employ other mechanisms, such than a monetary penalty, to discourage users from lying about contract outcomes when it benefits them, that would be an intriguing open challenge raised in the paper.

\vspace{.5cm}

\item \textbf{Paper Title: \textit{MAS and Blockchain: Results from a Systematic Literature Review}} \\

The creation of intelligent distributed systems that manage sensitive data makes extensive use of the \ac{MAS} technology. The usage of \ac{BCT} for \ac{MAS}  is encouraged by current trends in order to promote accountability and trustworthy connections. The researchers explained that as most of these techniques have just recently begun to investigate the subject, it is important to build a research road map and identify any relevant scientific or technological hurdles.

\vspace{.5cm}

This paper\cite{literature} includes a thorough literature evaluation of trials using \ac{MAS} and \ac{BCT} as conciliatory remedies as the first required step toward achieving this aim. The authors examined the reasons, presumptions, prerequisites, characteristics, and limits offered in the current state of the art in an effort to give a thorough review of their application fields. They also lay out their vision for how \ac{MAS} and \ac{BCT} may be coupled in various application situations while noting upcoming hurdles.
\end{itemize}

Understanding how applications employing \ac{MAS} and smart contracts are deployed inside a supply chain is crucial to ascertain if smart contracts and agents can be used to build a fair and efficient dispute resolution procedure in a supply chain. 

\vspace{.5cm}

As mentioned in our \textit{chapter's third question}, some work has been done combining agents and blockchain, but we are conducting additional research in this area because the agents we are about to use are \ac{BDI} agents, which are a type of software agent that can make autonomous decisions based on its beliefs about the environment, desires or goals, and intentions to achieve those goals. \ac{BDI} agents can assist with supply chain management by arranging the flow of commodities, monitoring inventory levels, and negotiating with suppliers and consumers. They may also be used to help with decision making, such as identifying the best suppliers to work with and identifying potential supply chain difficulties or problems.

\vspace{.5cm}

The next chapters attempt to construct our own application utilizing solidity-written smart contracts with JASON BDI agents to investigate if \ac{BDI} agents can enhance the collaboration, coordination and negotiation between supply chain members and if it can be used with smart contracts to manage unforeseen events and contingencies in supply chain.