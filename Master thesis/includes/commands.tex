% language constants
\def\english{english}
\def\german{german}
\def\ngerman{ngerman}
\def\true{true}
\def\false{false}

% DOCUMENT LAYOUT SETUP (Encoding, Language, Fonts, Page Setup) 
% 
\usepackage[utf8]{inputenc}	% Encoding
\usepackage{fontspec} % New font management (requires XeLaTeX)
\usepackage{lmodern} % Use (improved) font Latin Modern
% Custom Fonts
\linespread{1.05} % Adjust line spacing, depending on what font you use
\setmainfont{Lora}[
	Path 			= includes/fonts/Lora/ ,
	Extension		= .ttf ,
	UprightFont		= *-Regular ,
	BoldFont		= *-Bold ,
	ItalicFont		= *-Italic ,
	BoldItalicFont	= *-BoldItalic ,
	Scale			= .93 % Adjust font size
	 ]
\setsansfont{Lato}[
	Path 			= includes/fonts/Lato/ ,
	Extension		= .ttf ,
	UprightFont		= *-Regular ,
	BoldFont		= *-Bold ,
	ItalicFont		= *-Italic ,
	BoldItalicFont	= *-BoldItalic ]
\setmonofont{FiraCode}[
	Path 			= includes/fonts/FiraCode/ ,
	Extension		= .otf ,
	UprightFont		= *-Regular ,
	BoldFont		= *-Bold ,
	Scale			= MatchLowercase ]	

\ifx\doclanguage\english
		\usepackage[english]{babel}
\fi
\ifx\doclanguage\german
		\usepackage[german]{babel}
\fi
\ifx\doclanguage\ngerman
		\usepackage[ngerman]{babel}
\fi

\usepackage[left=30mm,right=20mm,top=30mm,bottom=40mm]{geometry} % Page setup

\usepackage[
	pdftoolbar=true,
	bookmarksopen,
	bookmarksnumbered=false,
	bookmarksopenlevel=1,
	pdfdisplaydoctitle,	
    colorlinks,
    citecolor=black,
    filecolor=black,
    linkcolor=black,
    urlcolor=black
]{hyperref} % Linking within PDF

% FORMAT CONTENT
\usepackage{graphicx} % Graphics options like scale

\usepackage[margin=10pt,font=small,labelfont=bf,labelsep=endash]{caption} % Nicer caption layouts

\usepackage{enumitem} % Customise enumeration formats
	% USAGE:
	%	\begin{enumerate}[label=\lbrack\arabic*\rbrack,start=1]
	%		\item Bla --> (creates [1] Bla)
	%	\end{enumerate}
	%
	%	\begin{enumerate}[label=\lbrack\arabic*\rbrack,resume]
	%		\item Bla --> (creates [2] Bla)
	%	\end{enumerate}
	
\usepackage{multicol} % Introduce multi column content
	% USAGE:
	% \begin{multicols}{<number>}
	% 	<multicolumn text>
	% \end{multicols}

\usepackage{url} % Insert clickable URLs
	% USAGE:
	% \url{http://www.google.com}

% PROVIDE CONTENT

\usepackage[printonlyused,withpage]{acronym} % Manage abbreviations
	% USAGE:
	% Create abbreviations in acronyms.tex this way...
	% 	\acro{POSF}{Partly Open Software Family}
	% 	\acroplural{POSF}[POSFs]{Partly Open Software Families}
	% Call abbreviations anywhere in the document this way...
	% 	\ac{abbr.}         % default:	full at first call, short (abbr.) else
	% 	\acs{abbr.}        % short: 	abbr.
	% 	\acf{abbr.}        % full: 	abbreviation (abbr.)
	% 	\acl{abbr.}        % long: 	abbreviation
	% for plural forms, add a -p: \acp, \acsp, \acfp, \aclp
	
\usepackage{blindtext} % Allow blind text
	% USAGE:
	% \blindtext[4] typesets blind text with 4 repetitions
	% \Blindtext[2][3] typesets blind text with 2 paragraphs and 3 rep.

%======================================================================== BIBLIOGRAPHY

\usepackage[
	backend=biber,
	style=alphabetic,
	maxbibnames=50,
	isbn=false,
	backref=true,
	backrefstyle=two+
]{biblatex} % User Biber and BibLaTeX

\bibliography{includes/bibliography}

%======================================================================== THEOREMS

\usepackage{mdframed} % Allows framed and coloured environments
\usepackage{thmtools} % Allows Theorem definitions and lists

\mdfdefinestyle{guidelinestyle}{
%  linecolor=black,					% STYLE 1 (boxed with grey title)
%  linewidth=.8pt,
%  frametitlerule=true,
%  frametitlefont=\sffamily\bfseries,
%  frametitlebackgroundcolor=codebackground,
%  innertopmargin=\topskip,
  frametitlealignment=\center,		% STYLE 2 (lines top, separator, bottom)
  frametitlefont=\normalfont,
  frametitlerule=true,
  rightline=false,
  leftline=false,
  linewidth=.8pt,
  frametitlerulewidth=.4pt
}

\makeatletter
\define@key{thmdef}{mdthm}[{}]{%
\thmt@trytwice{\def\thmt@theoremdefiner{\mdtheorem[#1]}}{}}
\makeatother

\declaretheorem[
	mdthm={style=guidelinestyle},
	%numberwithin=chapter
]{definition}% <environment name>
\makeatletter
\def\ll@definition{% <environment name>
  \protect\numberline{\csname the\thmt@envname\endcsname}
  \ifx\@empty\thmt@shortoptarg
    \thmt@thmname
  \else
    \thmt@shortoptarg
  \fi}
\makeatother
	% USAGE:
	%\begin{definition}[<term>]
	%\label{def:<term>}
	%	<definition text>
	%\end{definition}
	
\declaretheorem[
	mdthm={style=guidelinestyle},
	%numberwithin=chapter
]{formula}% <environment name>
\makeatletter
\def\ll@formula{% <environment name>
  \protect\numberline{\csname the\thmt@envname\endcsname}
  \ifx\@empty\thmt@shortoptarg
    \thmt@thmname
  \else
    \thmt@shortoptarg
  \fi}
\makeatother
	% USAGE:
	%\begin{formula}[<term>]
	%\label{for:<term>}
	%	<formula text>
	%\end{formula}


%======================================================================== LISTINGS

\usepackage{listings} % add source code inclusion opportunity

\lstset{
	basicstyle=\footnotesize\ttfamily,
	commentstyle=\color{tudgreen2}\ttfamily,
	keywordstyle=\bfseries\color{tudblue2},
	captionpos=b,
	language=Java,
	morekeywords={Set, Group, Feature, FeatureModel, Constraint}, 
	numbers=left,  
	numberstyle=\bfseries\ttfamily,
	stepnumber=1, 
	numbersep=5pt,
	tabsize=2,
	texcl=false,
	mathescape=true,
	frame=single,
	framexleftmargin=12pt,
	framextopmargin=1pt,
	framexbottommargin=1pt,
	framexrightmargin=4pt,
	backgroundcolor=\color{codebackground},
	showspaces=false,
	showstringspaces=false,
	numberbychapter=true
}
	% USAGE:
	% inline:
	%	\begin{lstlisting}[caption=<caption>, label=lst:<label>]
	%		<YOUR CODE HERE>
	%	\end{lstlisting}
	% include:
	%	\lstinputlisting[language=Python]{source_filename.py}

%======================================================================== COLOURS

\usepackage[usenames, dvipsnames, svgnames, table]{xcolor} % Manage colours

\xdefinecolor{tudblue}{HTML}{0B2A51}
\xdefinecolor{tudblue2}{HTML}{0059A3}
\xdefinecolor{tudpurple}{HTML}{51297F}
\xdefinecolor{tudpurple2}{HTML}{811A78}
\xdefinecolor{tudgreen}{HTML}{007A47}
\xdefinecolor{tudgreen2}{HTML}{22AD36}
\xdefinecolor{tudorange}{HTML}{E87B14}
\xdefinecolor{codebackground}{rgb}{0.96, 0.96, 0.96}
	% USAGE:
	% \textcolor{tudblue}

%======================================================================== TODOs
%-TODO Tag with optional comment
%(#1: comment) #2: affected region
\newcommand{\todo}[2][TODO]{\textbf{\textcolor{tudpurple2}{[#1]} \textcolor{tudorange}{#2}}}
%-ENHANCEMENT Tag
\newcommand{\enhancement}[1]{\textcolor{tudpurple2}{[Enhancement]} \textcolor{tudblue2}{#1}}
	% USAGE:
	% \todo{This is a sandbox todo.}
	% \todo[Wow]{This as well with own labeling.}
	% \enhancement{This is an enhancement.}




